\documentclass[10pt]{article}
%	options include 12pt or 11pt or 10pt
%	classes include article, report, book, letter, thesis

\usepackage[a4paper,bindingoffset=0.2in,%
left=1in,right=1in,top=0.2in,bottom=0.3in,%
footskip=.15in]{geometry}

\usepackage[T1]{fontenc}
\usepackage[polish]{babel}
\usepackage[utf8]{inputenc}
\usepackage{lmodern}
\usepackage{pgfplots}
\usepackage{graphicx}
\usepackage{amsmath}
\usepackage{subcaption}
\usepackage{multirow}
\usepackage{hyperref}
\usepackage{mathtools}
\newtheorem{hip}{Hipoteza}
\newtheorem{que}{Pytanie}
\newtheorem{wn}{Wniosek}
\newtheorem{wyd}{Zadanie}
\title{Algorytmy numeryczne}
\author{Zadanie 4 \\ Dawid Bińkuś \& Oskar Bir \& Mateusz Małecki\\grupa 1 tester-programista}
\date{13 Styczeń 2019}
\newcommand{\expnumber}[2]{{#1}\mathrm{e}{#2}}
\begin{document}
\maketitle 
\section{Aproksymacja}
Sprawozdanie prezentuje analizę aproksymacji dla problemu określonego w zadaniu 3.
W tym celu, zastosowana została aproksymacja dla metod testowanych w zadaniu 3:
\begin{itemize}
	\item Metoda Gaussa (PG) - wielomian 3-go stopnia,
	\item Metoda Gaussa z drobną optymalizacją dla macierzy rzadkich (SPG) - wielomian 2-go stopnia,
	\item Metoda Gaussa-Seidela (GS) przy założonej dokładności 1e-10 - wielomian 2-go stopnia,
	\\\\Oraz dodatkowo:
	\begin{itemize}
	\item Metoda zaimplementowana w oparciu o macierze rzadkie (S) - wielomian 1 stopnia (wykonane za pomocą LUDecomposition z biblioteki Apache Commons Math\footnote{\url{http://commons.apache.org/proper/commons-math/javadocs/api-3.6/overview-summary.html}}) - wariant z użyciem własnego typu danych (SparseFieldMatrix),
	\item wariant z użyciem tablicy double (DS) (OpenMapRealMatrix)
	\end{itemize}

\end{itemize}
Program na potrzeby analizy problemu został napisany w języku Java.
Ilość agentów oznaczana jest jako $N$.
\section{Próbka pomiarów czasu}
\subsection{Zakres testów}
Na potrzeby wyliczenia funkcji aproksymacyjnej dla każdej z metod, wykonane zostały testy dla $N = 15,16,...,60$, co przy $N = 60$ odpowiada liczbie około 2000 równań. Dla mniejszej ilości agentów testy wydajnościowe zostały wykonane kilka razy dla uśredniania wyniku.
\subsection{Wyniki}
Wyniki zostały zamieszczone w pliku \href{run:./files/wyniki.csv}{csv}.
\section{Aproksymacja średniokwadratowa dyskretna}
Za pomocą aproksymacji średniokwadratowej dyskretnej, wygenerowane zostały wielomiany dla każdego typu pomiarów. Prezentują się one w sposób następujący:
\begin{enumerate}
	\item Generowanie macierzy:
	\begin{enumerate}
		\item $f(x)=(\expnumber{2.3350439175615983}{-11})x^3 -(\expnumber{3.73562735299031}{-8}) x^2 + (\expnumber{3.927924861861045}{-5}) x^1 -0.006632202623758287$ dla PG,
		\item $f(x)=0.026544550834587136 x^1 -12.310426442549858$ dla S (wariant dla wielomianu stopnia 1, $m=1$),
		\item $(f(x)=\expnumber{1.1160411337156506}{-8})x^3 - (\expnumber{1.138102839915528}{-5})x^2 + 0.006355108980340516 x^1 \\-0.9349542064725634$ dla S (wariant dla wielomianu stopnia 3, $m=3$),
		\item $f(x)=0.005862735639798491x^1 -2.8318326460105085$ dla DS
		\item $f(x)=(\expnumber{3.905262387830365}{-8})x^2 -(\expnumber{2.5264926661242525}{-5})x^1 + 0.006283510185750934 x^0$ dla SPG,
		\item $f(x)=(\expnumber{5.486623477461811}{-8})x^2 - (\expnumber{5.4389374355751846}{-5}) x^1 + 0.014501100847044942$ dla GS.
	\end{enumerate}
	\item Rozwiązywanie układu równań:
	\begin{enumerate}
		\item $g(x)=(\expnumber{2.1592223868228134}{-8})x^3 - (\expnumber{3.716638790505861}{-5})x^2 + 0.023672650455756852x^1 \\-3.8108135102259824$ dla PG,
		\item $g(x)=(\expnumber{1.583603905695412}{-4})x^1 -0.06429601760699291$ dla S,
		\item $g(x)=(\expnumber{2.0353485741938283}{-5}) x^1,-0.008158601334700392$ dla DS,
		\item $g(x)=(\expnumber{1.3001654605220337}{-6})x^2 - 0.0014560130726117273x^1 + 0.3507906257127139$ dla SPG,
		\item $g(x)=(\expnumber{2.3048124134271384}{-5})x^2 - 0.020571995359005252x^1 + 4.514786569835568$ dla GS.
	\end{enumerate}
\end{enumerate}
\begin{figure}
	\caption{Wykresy reprezentujące porównanie funkcji aproksymacyjnej z rzeczywistymi wartościami danych metod. \label{Rys1}}
	\begin{subfigure}{0.5\textwidth}
		\includegraphics[width=\textwidth]{img/Exec/e1.png}
		\caption{ \label{Rys1a}}
	\end{subfigure}
	\begin{subfigure}{0.5\textwidth}
		\includegraphics[width=\textwidth]{img/Exec/e2.png}
		\caption{ \label{Rys1b}}
	\end{subfigure}
	\begin{subfigure}{0.5\textwidth}
		\includegraphics[width=\textwidth]{img/Exec/e3.png}
		\caption{ \label{Rys1c}}
	\end{subfigure}
	\begin{subfigure}{0.5\textwidth}
		\includegraphics[width=\textwidth]{img/Exec/www.png}
		\caption{ \label{Rys1d}}
	\end{subfigure}
	\begin{subfigure}{0.5\textwidth}
	\includegraphics[width=\textwidth]{img/Exec/exec_ds.png}
	\caption{ \label{Rys1j}}
	\end{subfigure}
	\begin{subfigure}{0.5\textwidth}
		\includegraphics[width=\textwidth]{img/Prep/wyk1.png}
		\caption{ \label{Rys1e}}
	\end{subfigure}
	\begin{subfigure}{0.5\textwidth}
		\includegraphics[width=\textwidth]{img/Prep/wyk2.png}
		\caption{ \label{Rys1f}}
	\end{subfigure}
		\begin{subfigure}{0.5\textwidth}
		\includegraphics[width=\textwidth]{img/Prep/wyk3.png}
		\caption{ \label{Rys1g}}
	\end{subfigure}
	\begin{subfigure}{0.5\textwidth}
		\includegraphics[width=\textwidth]{img/Prep/prep_ds.png}
		\caption{ \label{Rys1h}}
	\end{subfigure}
	\begin{subfigure}{0.5\textwidth}
		\includegraphics[width=\textwidth]{img/Prep/wyk5.png}
		\caption{ \label{Rys1i}}
\end{subfigure}
\end{figure}
\subsection{Poprawność uzyskanego rozwiązania}
\begin{center}
	\begin{tabular}{|l|l|l|}
		\hline
		\multicolumn{3}{ |c| }{Błąd aproksymacji} \\
		\hline
		Metoda & Wariant & Błąd aproksymacji[s]\\
		\hline
		\multirow{2}{*}{PG} & Obliczanie & $87.52296395468811$ \\
		& Generowanie & $0.0056371667465691345$ \\
		\hline
		\multirow{2}{*}{SPG} & Obliczanie & $2.932007861300984$\\
		& Generowanie & $0.0035357258834377552$\\
		\hline
		\multirow{2}{*}{GS} & Obliczanie & $204.23823436655252$ \\
		& Generowanie & $0.013028009281415509$\\
		\hline
		\multirow{3}{*}{S} & Obliczanie & $0.2257609275320777$ \\
		& Generowanie $m=1$ & $3727.8124913151905$\\
		& Generowanie $m=3$ & $26.142413480394573$\\
		\hline
		\multirow{2}{*}{DS} & Obliczanie & $\expnumber{8.484328415617502}{-4}$ \\
		& Generowanie & $208.62045033311406$ \\
		\hline
	\end{tabular}
\end{center}
Jak widać na załączonych wykresach \ref{Rys1} oraz tabeli powyżej prezentującej błąd aproksymacji, w większości przypadków funkcja aproksymująca poprawnie wylicza kolejne czasy wykonywania algorytmu. Wyjątkiem jest funkcja dla generowania w metodzie S. Wielomian stopnia pierwszego okazał się błędny dla tego zjawiska, dlatego wykorzystany został również wielomian stopnia trzeciego - który jak widać zwraca o wiele mniejszy błąd aproksymacji.
Pozostałe wyniki znajdują się w tolerancji błędu aproksymacji.
\section{Ekstrapolacja}
\begin{center}
	\begin{tabular}{|l|l|}
		\hline
		\multicolumn{2}{ |c| }{Wyliczony czas aproksymacji} \\
		\hline
		Metoda & f(100000) + g(100000) = czas[s]\\
		\hline
		PG & $2.1245904241142590 \cdot 10^7$ \\
		\hline
		SPG & $13244.4101182119$\\
		\hline
		GS & $228971.794504794$ \\
		\hline
		S $m=1$ & $2657.9164000555109$\\
		\hline
		S $m=3$ & $1.1047251400851820 \cdot 10^7$\\
		\hline
		DS &  $585.468921306697$ \\
		\hline
	\end{tabular}
\end{center}
\begin{wn}
Z załączonej wyżej tabeli wynika, że metoda DS jest najszybsza ze wszystkich rozważanych.\label{wn:1}
\end{wn}
\section{Próba obliczenia}
Na podstawie wniosku \ref{wn:1}, za najszybszą metodę została uznana metoda DS.
Ze względu na użycie biblioteki Commons Math, ograniczenia obiektu OpenMapRealMatrix (maksymalny rozmiar macierzy $ = ((X\cdot X) <=2147483647)$) oraz ograniczenia pamięciowe,konieczne było uruchomienie metody DS dla mniejszej ilości równań równej 12403. Rezultaty z próby obliczenia zostały zamieszczone w tabeli poniżej:
\begin{center}
	\begin{tabular}{|l|l|}
	\hline
	\multicolumn{2}{ |c| }{Podsumowanie wyników} \\
	\hline
	Metoda & Wartość [s]\\
	\hline
	DS - aproksymacja & $70.12796317673274$\\
	\hline
	DS - rzeczywisty czas & $1784.1351$\\
	\hline
\end{tabular}
\end{center}
Po przekroczeniu czasu pięciokrotnie, program znajdował się na etapie inicjacji solvera z biblioteki LUDecomposition, tj. obliczaniu dekompozycji macierzy.
\begin{wn}
Porównanie rzeczywistego czasu wykonania oraz wartości funkcji aproksymacyjnej dowodzi niedoskonałości aproksymacji.\label{wn:2}
\end{wn}

\section{Podział pracy}
\centering
\begin{tabular}{| p{4.4cm} | p{4.4cm} | p{4.4cm} |}
	\hline
	\textbf{Dawid Bińkuś} & \textbf{Oskar Bir} & \textbf{Mateusz Małecki} \\ \hline
	Praca nad strukturą projektu. & Przygotowanie testów i ich uruchomienie & Przygotowanie funkcji do agregowania danych \\ \hline
	Przygotowanie sprawozdania & Przygotowanie wykresów końcowych & Implementacja aproksymacji sredniokwadratowej dyskretnej\\ \hline
	Implementacja metod S i DS przy użyciu biblioteki & Implementacja aproksymacji sredniokwadratowej dyskretnej & Praca nad strukturą projektu \\ \hline	
	\end{tabular}
\end{document}