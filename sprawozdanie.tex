\documentclass[10pt]{article}
%	options include 12pt or 11pt or 10pt
%	classes include article, report, book, letter, thesis

\usepackage[a4paper,bindingoffset=0.2in,%
left=1in,right=1in,top=0.2in,bottom=0.3in,%
footskip=.15in]{geometry}

\usepackage[T1]{fontenc}
\usepackage[polish]{babel}
\usepackage[utf8]{inputenc}
\usepackage{lmodern}
\usepackage{pgfplots}
\usepackage{graphicx}
\usepackage{amsmath}
\usepackage{subcaption}
\usepackage{multirow}
\usepackage{hyperref}
\usepackage{mathtools}
\newtheorem{hip}{Hipoteza}
\newtheorem{que}{Pytanie}
\newtheorem{wn}{Wniosek}
\newtheorem{wyd}{Zadanie}
\title{Algorytmy numeryczne}
\author{Zadanie 4 \\ Dawid Bińkuś \& Oskar Bir \& Mateusz Małecki\\grupa 1 tester-programista}
\date{13 Styczeń 2019}

\begin{document}
\maketitle 
\section{Aproksymacja}
Sprawozdanie prezentuje analizę aproksymacji dla problemu określonego w zadaniu 3.
W tym celu, zastosowana została aproksymacja dla metod testowanych w zadaniu 3:
\begin{itemize}
	\item Metoda Gaussa (PG) - wielomian 3-go stopnia,
	\item Metoda Gaussa z drobną optymalizacją dla macierzy rzadkich (SPG) - wielomian 2-go stopnia,
	\item Metoda Gaussa-Seidela (GS) przy założonej dokładności 1e-10 - wielomian 2-go stopnia,
	\\\\Oraz dodatkowo:
	\item Metoda zaimplementowana w oparciu o macierze rzadkie (S) - wielomian 1 stopnia (wykonane za pomocą LUDecomposition z biblioteki Apache Commons Math\footnote{\url{http://commons.apache.org/proper/commons-math/javadocs/api-3.6/overview-summary.html}})
\end{itemize}

\section{Podział pracy}
\centering
\begin{tabular}{| p{4.4cm} | p{4.4cm} | p{4.4cm} |}
	\hline
	\textbf{Dawid Bińkuś} & \textbf{Oskar Bir} & \textbf{Mateusz Małecki} \\ \hline
	Praca nad strukturą projektu. & Analiza algorytmu Gaussa oraz implementacja wariantu G & Implementacja typu własnej precyzji \\ \hline
	Przygotowanie sprawozdania & Przygotowanie testów i ich uruchomienie & Operacje na macierzach\\ \hline
	Implementacja algorytmu Gaussa w wariantach PG i FG & Analiza danych oraz określenie czasu pracy typu Fraction & Praca nad strukturą projektu \\ \hline
	Implementacja generycznej klasy MyMatrix & Przygotowanie wykresów końcowych & Implementacja generycznej klasy MyMatrix \\ \hline	
	\end{tabular}
\end{document}